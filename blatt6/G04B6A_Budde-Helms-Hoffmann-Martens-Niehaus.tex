\documentclass[ngerman]{fbi-aufgabenblatt}
\usepackage{amsmath}
\usepackage{amsfonts}
\usepackage{amssymb}
\usepackage{paralist}
%\usepackage{blockgraph}

\usepackage{listings}
\usepackage{color}
\usepackage{textcomp}

\definecolor{mygreen}{rgb}{0,0.6,0}
\definecolor{mygray}{rgb}{0.5,0.5,0.5}
\definecolor{mymauve}{rgb}{0.58,0,0.82}

\lstset{ %
  backgroundcolor=\color{white},   % choose the background color
  basicstyle=\footnotesize,        % size of fonts used for the code
  breaklines=true,                 % automatic line breaking only at whitespace
  captionpos=b,                    % sets the caption-position to bottom
  commentstyle=\color{mygreen},    % comment style
  escapeinside={\%*}{*)},          % if you want to add LaTeX within your code
  keywordstyle=\color{blue},       % keyword style
  stringstyle=\color{mymauve},     % string literal style
}

% Folgende Angaben bitte Anpassen !!!

\renewcommand{\Aufgabenblatt}{6}
\renewcommand{\Gruppe}{G04}
\renewcommand{\KleinGruppe}{A}
\renewcommand{\Teilnehmer}{Budde, Helms, Hoffmann, Martens, Niehaus}

\begin{document}

\aufgabe{Speicherverwaltung}

\subsection{Wie groß ist eine Seite?}

Bei 16-bit virtuellen Adressen ($\hat =$ 64 KiB) und 16 Seiten ist eine Seite entsprechend 4 KiB groß.

\subsection{typischer Seitentabelleneintrag}

Nach Tanenbaum\footnote{Andrew S. Tanenbaum, Modern Operating Systems, 3rd Edition, 3.3.2 Structure of a Page Table Entry}
besteht eine typische Seitentabelle aus:
\begin{itemize}
 \item Page Frame Number - welche die jeweilige Seite eindeutig bestimmt.
 \item ein Present/absent bit - welches Angibt ob die Seite z.Zt im Speicher liegt oder ausgelagert wurde.
 \item Protection bit(s) - ein oder mehrere bit(s), welche einfache (rw) oder komplexere (rwx) Zugriffsrechte regeln.
 \item Modified and Referenced bits - welche die Nutzung der Seite dokumentieren. Z.B. wird bei einer Änderung das
 Modified bit gesetzt um festzustellen ob die Seite zurück auf die Festplatte geschrieben werden muss oder verworfen werden kann.\\
 Das Referenced bit gibt an ob eine Reference auf diese Seite gehalten wird. Dies ermöglicht dem Betriebssystem bessere Entscheidungen
 bei einem Pagefault\footnote{Seite befindet sich z.Zt. nicht im Arbeitsspeicher.} zu treffen.
 \item Caching bit - gibt an ob Puffern für diese Seite erlaubt ist. Dies kann wichtig sein wenn die Seite etwas anderes als den
 Arbeitsspeicher referenziert, z.B. eine Festplatte oder ein Bandlaufwerk.
\end{itemize}

%Die optimale Seitengröße kann nach Tanenbaum\footnote{Andrew S. Tanenbaum, Modern Operating Systems, 3rd Edition, pg. 220} durch $$p = \sqrt{2se}$$
%ermittelt werden. Wobei $s:=$ Durchschnittliche Prozessgröße und $e:=$ benötigter Speicher pro Seite in der Seitentabelle.

\subsection{welche physikalische Adresse?}

\subsubitem{i)} \texttt{0x5fe8} $=24.552$ liegt im Bereich 5 (20.480 bis 24.575) \\
dort ist der Pageframe $001_2 = 1$ angegeben. \\
Wir mappen also auf $1 \cdot 4096 + (24.552 - 20.480) = 8.169 =$ \texttt{0x1fe8}
\subsubitem{ii)} \texttt{0xfeee} $= 65.262$ liegt im Bereich 15 (61.440 bis 65.535)\\
dort ist der Pageframe $000_2 = 0$ angegeben, aber das Present Bit ist nicht gesetzt.\\
Die entsprechende Seite muss also erst in den Arbeitsspeicher geladen werden.
\subsubitem{iii)} \texttt{0xa470} $=42.096$ liegt im Bereich 10 (40.960 bis 45.055) \\
dort ist der Pageframe $000_2 = 0$ angegeben. \\
Wir mappen also auf $0 \cdot 4096 + (42.096 - 40.960) = 1.136 =$ \texttt{0x0470}
\subsubitem{iv)} \texttt{0x0101} $=257$ liegt im Bereicht 0 (0 bis 4.095) \\
dort ist der Pageframe $101_2 = 5$ angegeben. \\
Wir mappen also auf $5 \cdot 4096 + (257 - 0) = 20737 = $ \texttt{0x5101}


\subsection{optimale Seitengröße I}

Für kleine Seiten spricht:
\begin{itemize}
 \item Zufällig gewählte Daten passen in der Regel nicht genau in ganzzahlige Seiten.
 Im Schnitt bleibt die Hälfte der letzten Seite leer und ist entsprechend verschwendet.
 Dies nennt man interne Fragmentierung. Bei $n$ Bereichen im Arbeitsspeicher und $p$ byte 
 Seitengröße sind $\frac{np}{2}$ byte verschwendet.
 \item Ein weiterer Faktor bei der Verschwendung sind Programme. Jedes Programm benötigt mindestens
 eine volle Seite unabhängig von der eigentlichen größe des Programms. Auch hier gilt: je kleiner
 die Seitengröße gewählt ist desto geringer ist der verschwendete Platz im Speicher.
\end{itemize}
Für große Seiten hingegen spricht:
\begin{itemize}
 \item Große Programme bzw. großer Arbeitsspeicher hingegen sprechen für große Seiten.
 Ein 32KiB Programm benötigt lediglich vier Seiten á 8 KiB, aber 64 Seiten bei 512 byte größe pro Seite. 
 
 Viele Seiten sind aus 2 Gründen negativ. Zum einen wird der Arbeitsspeicher in der Regel Seitenweise
 befüllt was viel Zeit in Anspruch nimmt wenn viele Seiten geladen werden sollen.
 Zum anderen benötigen kleine Seiten bei großen Arbeitsspeicher eine große Pagetable. Je nach System
 kann es nötig sein das die Pagetable für jede Bewegung einer Seite in ein CPU Register geladen werden
 muss.

\end{itemize}
Vgl. jeweils Tannenbaum.\footnote[3]{Andrew S. Tanenbaum, Modern Operating Systems, 3rd Edition, pg. 219 ff.}

\subsection{optimale Seitengröße II}

Bei den gegebenen Werten\footnote{$p =4$MiB durchschnittl. Prozessgröße und $L=8$B Länge eines Pagetableeintrags.}
ist die Anzahl an Seiten $S$ pro Prozeß in etwa $S = \frac pA$, wobei $A$ die Seitengröße ist. Jeder Prozeß benötigt
also im Schnitt $\frac {Lp}A$ byte der Pagetable. Verlust an Speicher durch 
sie letzte Seite ist $\frac A 2$. Wir haben also insgesamt einen overhead $o$ von:
$$o= \frac {Lp} A + \frac A 2$$
$\frac {Lp}A$ ist groß bei kleiner Seitengröße. Bei großer Seitengröße allerdings wird $\frac A2$ groß. Für die optimale
Seitengröße $A_{opt.}$ ergibt sich daraus:
$$A_{opt.} = \sqrt{2pL} = \sqrt{2\cdot (1024^2\text{Byte})(8\text{B})} = \sqrt{16\cdot 1024^2 \text{B}^2} = 4\cdot 1024 \text{B} = 4\text{KiB} $$

Vgl. auch hier je Tannenbaum.\footnotemark[3]

\aufgabe{Seitenersetzungsalorithmen}

\subsection{Seitenspeicher}

\subsubitem{optimaler Seitenersetzungsalorithmus:}

 \begin{tabular}{c||c|c|c|c|c|c|c|c|c|c|c|c|c|c|c}
 $t$ 		& 1 	& 2 	& 3 	& 4 	& 5 	& 6 	& 7 	& 8 	& 9 	& 10 			& 11 		& 12 		& 13 		& 14 		& 15 \\\hline 
 Seite 		& 1 	& 2 	& 3 	& 4 	& 5 	& 6 	& 1 	& 3 	& 1 	&  6 			&  3 		&  5 		&  4 		&  2 		&  1 \\\hline 
 Fault		& X 	& X 	& X 	& X 	& X 	& X 	& . 	& . 	& . 	&  . 			&  . 		&  X 		&  X 		&  X 		&  X \\\hline 
 Seiten		& 1 	& 1 	& 1 	& 1 	& 1 	& 1 	& 1 	& 1 	& 1 	&  1 			&  1 		&  1 		&  1 		&  1 		&  1 \\
 im		&   	& 2 	& 2 	& 4 	& 5 	& 6 	& 6 	& 6 	& 6 	&  6 			&  6 		&  5 		&  5 		&  5 		&  5 \\
 Speicher	&   	&   	& 3 	& 3 	& 3 	& 3 	& 3 	& 3 	& 3 	&  3 			&  3 		&  3 		&  4 		&  2 		&  2 \\\hline
 $p_i^*$ der	& $5_1$ & $4_1$ & $3_1$ & $1_1$ & $1_1$	& $0_1$ & $1_1$ & $0_1$ & $5_1$ &  $4_1$ 		&  $3_1$ 	&  $2_1$ 	&  $1_1$ 	&  $0_1$ 	&  $\infty_1$ \\
 jeweiligen	&   	& $11_2$& $10_2$& $9_4$	& $6_5$ & $3_6$ & $2_6$ & $1_6$ & $0_6$ &  $\infty_6^{**}$ 	&  $\infty_6$ 	&  $\infty_5$ 	&  $\infty_5$ 	& $\infty_5$ 	&  $\infty_5$ \\
 Seite		&   	&   	& $4_3$	& $3_3$	& $2_3$ & $1_3$ & $0_3$ & $2_3$ & $1_3$ &  $0_3$ 		&  $\infty_3$ 	&  $\infty_3$ 	&  $\infty_4$ 	& $\infty_2$ 	&  $\infty_2$ \\
 \end{tabular}
 
 $*$ das $i$ von $p_i$ bezieht sich auf die Seite, der Wert von $p$ auf die Anzahl der Befehle die noch
 auszuführen sind bis die Seite wieder benutzt wird.
 
 $**$ Das Zeichen $\infty$ wird in diesem Zusammenhang benutzt um ``nicht in vorhersehbarer Zeit'' auszudrücken.
 \subsubitem{LRU-Algorithmus:}

 \begin{tabular}{c||c|c|c|c|c|c|c|c|c|c|c|c|c|c|c}
 $t$ 		& 1 & 2 & 3 & 4 & 5 & 6 & 7 & 8 & 9 & 10 & 11 & 12 & 13 & 14 & 15 \\\hline 
 Seite 		& 1 & 2 & 3 & 4 & 5 & 6 & 1 & 3 & 1 &  6 &  3 &  5 &  4 &  2 &  1 \\\hline 
 Fault		& X & X & X & X & X & X & X & X & . &  . &  . &  X &  X &  X &  X \\\hline 
 Seiten		& 1 & 1 & 1 & 4 & 4 & 4 & 1 & 1 & 1 &  1 &  1 &  1 &  1 &  2 &  2 \\
 im		&   & 2 & 2 & 2 & 5 & 5 & 5 & 3 & 3 &  3 &  3 &  5 &  5 &  5 &  1 \\
 Speicher	&   &   & 3 & 3 & 3 & 6 & 6 & 6 & 6 &  6 &  6 &  6 &  4 &  4 &  4 \\\hline
 Liste		& 1 & 2 & 3 & 4 & 5 & 6 & 1 & 3 & 1 &  6 &  1 &  5 &  4 &  2 &  1 \\
 der ehem.	&   & 1 & 2 & 3 & 4 & 5 & 6 & 1 & 3 &  1 &  6 &  1 &  5 &  4 &  2 \\
 Seiten$^{***}$	&   &   & 1 & 2 & 3 & 4 & 5 & 6 & 6 &  3 &  3 &  6 &  1 &  5 &  4 \\
 \end{tabular}
 
 $***$ älteste unten, jüngste oben.

 \subsection{optimaler Seitenersetzungsalorithmus}
 
 Der optimale Seitenersetzungsalorithmus kommt in realen Betriebssystem nicht zum Einsatz,
 da in der Regel nicht bekannt ist wann welche Seite wieder gebraucht wird. Die Reihenfolge in der Prozesse
 ausgeführt werden ist in der Regel nicht deterministisch. Die Reihenfolge der Ausführung der Prozesse und
 damit auch der Abstand zur nächsten Verwendung einer bestimmten Seite kann sich ständig ändern.

 \subsection{Last Recently Used?}
 
 Last recently used übersetzt sich zu ``zuletzt kurz zuvor genutzt'' im Gegensatz zu least recently used
 (``am wenigsten kurz zuvor genutzt''). Das würde als Seitenersetzungsalorithmus bedeuten das genau die
 Seite die eben noch benutzt wurde aus dem Speicher entfernt wird falls ein Pagefault auftritt.
 Das erscheint intuitiv unsinnig, da Programme in der Regel längere Zeit ausgeführt werden. ``Last recently
 used'' würde nur Sinn machen jede Seite nur sehr selten - idealerweise nur einmal genutzt - wird.
 
\aufgabe{Synchronisation}

\begin{enumerate}
	\item

\begin{verbatim}
semaphore W=1, Mutex=1;
NumberOfActiveReaders=0;
 
processWriter()
{
    wait(W);
    // Writing is done
    signal(W);
}
 
processReader()
{
    wait(Mutex);
        NumberOfActiveReaders++;
        if (NumberOfActiveReaders == 1)
            wait(W);
    signal(Mutex);
    // Do the Reading
    // (Critical Section Area)
    wait(Mutex);
        NumberOfActiveReaders--;
        if (NumberOfActiveReaders == 0)
            signal(W);
    signal(Mutex);
}

\end{verbatim}

\item 
Write-Prozesse werden benachteiligt, da theoretisch beliebig viele Read-Prozesse ankommen können und der Write-Prozess nie ausgeführt wird. Eine Lösung wäre sicherzustellen, dass kein Prozess aussterben darf (weder Write- noch Read-Prozesse).

\end{enumerate}


\end{document}
