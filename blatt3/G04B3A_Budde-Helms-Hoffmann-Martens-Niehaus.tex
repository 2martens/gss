\documentclass[ngerman]{fbi-aufgabenblatt}
\usepackage{amsmath}
\usepackage{amsfonts}
\usepackage{amssymb}
\usepackage{paralist}

\usepackage{listings}
\usepackage{color}

\definecolor{mygreen}{rgb}{0,0.6,0}
\definecolor{mygray}{rgb}{0.5,0.5,0.5}
\definecolor{mymauve}{rgb}{0.58,0,0.82}

\lstset{ %
  backgroundcolor=\color{white},   % choose the background color
  basicstyle=\footnotesize,        % size of fonts used for the code
  breaklines=true,                 % automatic line breaking only at whitespace
  captionpos=b,                    % sets the caption-position to bottom
  commentstyle=\color{mygreen},    % comment style
  escapeinside={\%*}{*)},          % if you want to add LaTeX within your code
  keywordstyle=\color{blue},       % keyword style
  stringstyle=\color{mymauve},     % string literal style
}

% Folgende Angaben bitte Anpassen !!!

\renewcommand{\Aufgabenblatt}{3}
\renewcommand{\Gruppe}{G04}
\renewcommand{\KleinGruppe}{A}
\renewcommand{\Teilnehmer}{Budde, Helms, Hoffmann, Martens, Niehaus}

\begin{document}

% solange Aufgaben 1 und 2 nicht bearbeitet wurden
% täuscht LaTex vor, dass bereits Section 1 und 2 gesetzt wurden (Aufgabe=Section)
%\setcounter{section}{2}

\aufgabe{Rechnersicherheit}

	\subsection{Zugangs- und Zugriffskontrolle}
	
	\begin{enumerate}
		\item
		Zugangskontrolle: 
		\begin{itemize}
			\item nur mit berechtigten Partnern weiter kommunizieren
			\item verhindern von unbefugter Inanspruchnahme von Betriebsmitteln
		\end{itemize}
		Zugriffskontrolle: 
		\begin{itemize}
			\item Ausführen von Operation auf Objekt nur möglich, wenn das Subjekt das Recht dazu hat.
		\end{itemize}

		\item
		Es ist sinnvoll ein System nur mit einer Zugangskontrolle auszustatten. Dies kann zum Beispiel dann eingesetzt werden, wenn alle Personen, die durch die Zugangskontrolle kommen, Zugriff haben sollen. Konkret ist dies beispielsweise bei den Gedächtnisprotokollen der Fachschaft Informatik der Fall. Die Zugangskontrolle besteht darin, dass nur IP-Adressen aus dem Bereich der Informatik sich die Protokolle ansehen können. Besitzt man jedoch eine dieser Adressen, dann gibt es keine weiteren Einschränkungen.
		
		\item 
		Die Zugriffskontrolle setzt die Zugangskontrolle voraus, da im Rahmen der Zugangskontrolle eine Identifikation der Person stattfindet. Ohne diese Identifikation kann nicht geprüft werden, ob eine Person überhaupt rechtmäßig zugreifen kann. Ein simples Beispiel dazu sind Forensysteme. Auch wenn man ein Konto besitzt, so hat man als Gast nicht die Rechte eines angemeldeten Nutzers, denn das System kann einen nicht eindeutig identifizieren und mit den Rechtetabellen abgleichen.
		Sobald man jedoch angemeldet ist, kann man vom System eindeutig identifiziert werden (über Benutzernamen und/oder E-Mail-Adresse) und ein Rechteabgleich ist daher möglich.
		
		\item
		Die Zugangskontrolle wird hier durch den Benutzer, der den Link weitergibt, durchgeführt. Wenn er einer Person nicht vertraut, wird er dieser nicht den Link geben. Theoretisch lässt sich dies auch transitiv fortsetzen. Der bewusste Akt der Weitergabe des Links entspricht der Zugangskontrolle. Jeder kann somit mit Kenntnis des Links auf den Ordner zugreifen. Eine Zugriffskontrolle findet nicht statt.
		
		Wird der Ordner jedoch freigegeben, dann erlaubt der Link keinen Zugang zum Ordner, sondern ermöglicht es lediglich sich den Zugriff zum Ordner über einen eigenen Account zu sichern. Die Zugangskontrolle findet hierbei durch den Login statt und die Zugriffskontrolle durch die Berechtigung in Form der Freigabe.
	\end{enumerate}
	
\newpage 
\aufgabe{Timing-Attack}

	\texttt{Quelltext auch Online: https://gist.github.com/lino/7b685327bd779e9366ce}

	\lstinputlisting[language=Java]{code/src/TimingAttack.java}

	\subsection{}
	Siehe Quelltext

	\subsection{}

	Ein Timing-Angriff ist möglich, da sich die Laufzeit in 2 wichtigen Parametern des eingegebenen Passworts im Vergleich zum korrekten Passwort ändert:
	
	\begin{itemize}
		\item[i)] Hat ein eingegebenes Passwort die gleiche Länge, wie das erwartete korrekte Passwort, so werden die ersten beiden Passwortziffern verglichen. Ist dem nicht so gibt es sofort einen Return, die Laufzeit ist kürzer.
		\item[ii)] Mit jeder (von links aus gezählten) aufeinanderfolgenden korrekten Ziffer im Eingabepasswort wird ein weiterer Vergleich von der darauffolgenden Ziffer garantiert, also die Laufzeit verlängert.
	\end{itemize}

	\subsection{}

	Nach den in 2.2 beschriebenen Mechanismen der Laufzeitverlängerung bietet sich folgendes Vorgehen an:

	\begin{itemize}
		\item[i)] Man ermittelt die korrekte Passwortlänge. Hierzu wählt man ein beliebiges Zeichen, etwa $0$. Zuerst testet man das leere Passwort und erhöht dann, mit dem gewählten Zeichen solange die Länge des eingegebenen Passworts, bis ein signifikanter Laufzeitunterschied (Verlängerung) zur Vorgängereingabe erkannt wird. Denn nun wird mindestens die erste Eingabeziffer mit dem Korrekten Passwort abgeglichen.
		
		\item[ii)] Ist die Länge bekannt probiert man, in beliebiger Reihenfolge in der ersten Passwortziffer solange alle möglichen Eingaben aus, bis ein signifikanter Laufzeitunterschied festgestellt werden kann. Die Eingabe mit der längeren Laufzeit wird gewählt (ggf. kann ja auch die erste gewählte Eingabe hier $0$ korrekt sein).\\
Dieser Schritt wird für jede Passwortziffer wiederholt, bis alle Ziffern bekannt sind.
	\end{itemize}

	\newpage

	\subsection{}

	Beim TimingSafePasswordChecker wird auch bei unterschiedlichen Arraylängen stets eine bei gleichbleibender Passwort konstante Laufzeit angestrebt. Hierdurch sind Timingangriffe erschwert, auch wenn Zufallstreffer weiterhin möglich sind.

	Ausgabe des Vergleichs:

	\lstinputlisting[language=Java]{output.txt}

\aufgabe{Real-World Brute-Force Angriff}

	Zur Begutachtung des Sicherheitscodes wurden einige Test-PDFs\footnote{Beitrittserklärungen zum Chaos Computer Club e.V.} hochgeladen. Die zurückgegebenen Sicherheitscodes waren

	\begin{itemize}
		\item{\texttt{GSS14-DEMO-FDJL6-SL6JR-AN4KF-H74FL}}
		\item{\texttt{GSS14-DEMO-XAKEQ-LYYG9-3FRQX-CFPC6}}
		\item{\texttt{GSS14-DEMO-HM94M-THYZJ-H7FRD-QKHSA}}
		\end{itemize}

	Der Sicherheitscode hat eine statische Länge von 34 Zeichen. Der Prefix \texttt{GSS14-DEMO-} ist statisch, bleiben 23 Zeichen. Die drei Trennstriche sind ebenfalls statisch, bleiben 20 Zeichen. 

	Es werden lediglich Großbuchstaben ohne Umlaute und Zahlen verwendet, sodass unser Alphabet aus 36 Zeichen besteht.

	\begin{align*}
		\Sigma = \left\{A,B,C,D,E,F,G,H,I,J,K,L,M,N,O,P,Q,R,S,T,U,V,W,X,Y,Z,0,1,2,3,4,5,6,7,8,9\right\}
	\end{align*}

	Da schon in den drei Testcodes einzelne Zeichen doppelt vorkommen, lässt sich die Anzahl der möglichen Kombinationen durch ein Urnenmodell mit sortierter Ziehung mit Zurücklegen abbilden. Die Anzahl der möglichen Kombinationen aus $N$ Zahlen und $n$ Stellen ist mit $N^n$ definiert. Folglich gibt es

	\begin{align*}
		36^{20} = 1.3367*10^{31}
	\end{align*}

	Möglichkeiten. Bei 1000 Anfragen pro Sekunde könnte man im Schlimmsten Fall $1.3367*10^{28}$ Sekunden bzw. $4.2388*10^{17}$ Jahrtausende warten bis die gewünschte Kombination gefunden ist. Um einen Mittleren Wert zu finden, betrachten wir den Einbruchversuch als Bernoulli-Versuch. Derzeit gibt es 25 Abgaben, also 25 gütlige Schlüssel im System. Die Wahrscheinlichkeit einen Schlüssel zufällig zu erraten wäre $1.8702*10^{-30}$. 

	Die Anzahl der Versuche um in 50\% der Fälle einen von 25 Schlüsseln zu finden sei $n$

	\begin{align*}
		0.5 = n*\frac{25}{36^{20}}*(1-\frac{25}{36^{20}})^{n-1}
	\end{align*}

	Durch umstellen lässt sich ermitteln, dass n unendlich groß ist.

\end{document}
