\documentclass[ngerman]{fbi-aufgabenblatt}

% Folgende Angaben bitte Anpassen !!!

\renewcommand{\Aufgabenblatt}{1}
\renewcommand{\Gruppe}{G03}
\renewcommand{\KleinGruppe}{A}
\renewcommand{\Teilnehmer}{Budde, Helms, Hoffmann, Martens, Niehaus}

\begin{document}

% solange Aufgaben 1 und 2 nicht bearbeitet wurden
% täuscht LaTex vor, dass bereits Section 1 und 2 gesetzt wurden (Aufgabe=Section)
%\setcounter{section}{2}

\aufgabe{Grundlagen von Betriebssystemen}

% 1.a

\begin{enumerate}
	\item Eine Grundaufgabe eines Betriebssystems (BS) ist die Vermittlung zwischen Hard- und Software. Das BS koordiniert dabei über Treiber die Hardware und die Software kann über entsprechende Schnittstellen universell darauf zugreifen.
	Die zweite Grundaufgabe eines BS ist die Vermittlung zwischen Programmen. Mehrere Programme können sich in einem Deadlock befinden, welcher nur durch das BS lösbar ist.
	
	\item Das BS als Vermittler zwischen Programmen muss sich konkret darum kümmern, dass
	\begin{itemize}
		\item Programme nicht in einen Deadlock geraten und dass
		\item Programme nicht auf die selben Dateien bzw. auf den Speicherbereich von anderen Programmen zugreifen (Ausnahmen gibt es natürlich z.B. bei Virenscannern) können.
	\end{itemize}
	Das BS als Vermittler zwischen Hard- und Software muss sich konkret
darum kümmern, dass
	\begin{itemize}
		\item Programme Ton wiedergeben lassen können ohne konkrete Kenntnisse über die verwendete Hardware zu haben (Treiberverwaltung) und dass
		\item Hardware die angeschlossen wird (z.B. via USB) direkt der Software zur Verfügung steht.
	\end{itemize}
\end{enumerate}

\aufgabe{Prozesse und Threads}

\begin{enumerate}
	\item Ein Programm ist eine Folge von Anweisungen zur Bearbeitung von Daten oder der Lösung eines Problems.
\end{enumerate}
%\input{betriebssysteme}

\end{document}