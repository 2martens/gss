\documentclass[ngerman]{fbi-aufgabenblatt}

% Folgende Angaben bitte Anpassen !!!

\renewcommand{\Aufgabenblatt}{1}
\renewcommand{\Gruppe}{G01}
\renewcommand{\KleinGruppe}{A}
\renewcommand{\Teilnehmer}{Budde, Hoffmann, Martens, Niehaus}

\begin{document}

% solange Aufgaben 1 und 2 nicht bearbeitet wurden
% täuscht LaTex vor, dass bereits Section 1 und 2 gesetzt wurden (Aufgabe=Section)
\setcounter{section}{2}

\aufgabe{Angreifermodell}

% 3.1
\subsection{}

Ein Angreifermodell beschreibt einen Angreifer, gegen den ein Schutzmechanismus wirken soll. In dem Modell wird außerdem der Mechanismus beschrieben, der gegen den Angreifer wirken soll.

Der Angreifer selber wird mithilfe von vier Kriterien beschrieben. Dies ist zum Einen die Rolle. So kann ein Angreifer Outsider oder Insider sein. Für den Fall des Insiders gibt es diverse weitere Unterteilungen, wie beispielsweise Benutzer, Betreiber, Wartungsdienst, Produzent oder Entwerfer. Diese Rollen können auch kombiniert auftreten.

Außerdem wird die Verbreitung des Angreifers berücksichtigt. Die beschreibt die Stellen im System, an denen der Angreifer Informationen gewinnen oder Systemzustände verändern kann.

Desweiteren wird das Verhalten des Angreifers erwähnt. Ein Angreifer kann entweder passiv oder aktiv, beziehungsweise beobachtend oder verändernd sein.

Schließlich wird die Rechenkapazität des Angreifers in Betracht gezogen, die entweder unbeschränkt oder beschränkt sein kann. Ist sie unbeschränkt, so ist der Mechanismus, der gegen den Angreifer gerade noch wirkt, informationstheoretisch interessant. Wenn die Rechenkapazität beschränkt ist, so handelt es sich um einen komplexitätstheoretischen Mechanismus.

%\input{betriebssysteme}

\end{document}