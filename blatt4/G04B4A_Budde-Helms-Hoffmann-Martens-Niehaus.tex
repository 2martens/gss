\documentclass[ngerman]{fbi-aufgabenblatt}
\usepackage{amsmath}
\usepackage{amsfonts}
\usepackage{amssymb}
\usepackage{paralist}
\usepackage{blockgraph}

\usepackage{listings}
\usepackage{color}

\definecolor{mygreen}{rgb}{0,0.6,0}
\definecolor{mygray}{rgb}{0.5,0.5,0.5}
\definecolor{mymauve}{rgb}{0.58,0,0.82}

\lstset{ %
  backgroundcolor=\color{white},   % choose the background color
  basicstyle=\footnotesize,        % size of fonts used for the code
  breaklines=true,                 % automatic line breaking only at whitespace
  captionpos=b,                    % sets the caption-position to bottom
  commentstyle=\color{mygreen},    % comment style
  escapeinside={\%*}{*)},          % if you want to add LaTeX within your code
  keywordstyle=\color{blue},       % keyword style
  stringstyle=\color{mymauve},     % string literal style
}

% Folgende Angaben bitte Anpassen !!!

\renewcommand{\Aufgabenblatt}{4}
\renewcommand{\Gruppe}{G04}
\renewcommand{\KleinGruppe}{A}
\renewcommand{\Teilnehmer}{Budde, Helms, Hoffmann, Martens, Niehaus}

\begin{document}

\aufgabe{Scheduling-Algorithmen}

\begin{enumerate}

\item 
\begin{blockgraph}{30}{1}{0.4}
	% \bglabelxx erzeut Beschriftung der X-Achse an bestimmter Position
	\bglabelxx{0}
	\bglabelxx{5}
	\bglabelxx{10}
	\bglabelxx{15}
	\bglabelxx{20}
	\bglabelxx{25}
	\bglabelxx{30}

	% \bgblock erzeugt Block innerhalb des Graphen. Parameter:
	%    Y-Position (z.B. CPU), optional
	%    Beginn auf der X-Achse
	%    Ende auf der X-Achse
	%    Beschriftung
	\bgblock{1}{7}{$P_1$}
	\bgblock{8}{9}{$P_3$}
	\bgblock{10}{12}{$P_4$}
	\bgblock{13}{18}{$P_2$}
	\bgblock{19}{27}{$P_5$}

\end{blockgraph}

\item
\begin{blockgraph}{35}{1}{0.4}
	% \bglabelxx erzeut Beschriftung der X-Achse an bestimmter Position
	\bglabelxx{0}
	\bglabelxx{5}
	\bglabelxx{10}
	\bglabelxx{15}
	\bglabelxx{20}
	\bglabelxx{25}
	\bglabelxx{30}
	\bglabelxx{35}

	% \bgblock erzeugt Block innerhalb des Graphen. Parameter:
	%    Y-Position (z.B. CPU), optional
	%    Beginn auf der X-Achse
	%    Ende auf der X-Achse
	%    Beschriftung
	\bgblock{1}{5}{$P_1$}
	\bgblock{6}{7}{$P_3$}
	\bgblock{8}{10}{$P_4$}
	\bgblock{11}{13}{$P_5$}
	\bgblock{14}{16}{$P_2$}
	\bgblock{17}{19}{$P_1$}
	\bgblock{20}{22}{$P_5$}
	\bgblock{23}{25}{$P_2$}
	\bgblock{26}{28}{$P_5$}
	\bgblock{29}{30}{$P_2$}
	\bgblock{31}{33}{$P_5$}

\end{blockgraph}

\end{enumerate}

\aufgabe{Echtzeit \& Mehrprozessor-Scheduling}

\begin{enumerate}
	\item
	todo
	\item
	\begin{enumerate}
	
	\item 
	$ $\\
	\begin{blockgraph}{28}{1}{0.4}
		% \bglabelxx erzeut Beschriftung der X-Achse an bestimmter Position
		\bglabelxx{0}
		\bglabelxx{4}
		\bglabelxx{8}
		\bglabelxx{12}
		\bglabelxx{16}
		\bglabelxx{20}
		\bglabelxx{24}
		\bglabelxx{28}

		% \bgblock erzeugt Block innerhalb des Graphen. Parameter:
		%    Y-Position (z.B. CPU), optional
		%    Beginn auf der X-Achse
		%    Ende auf der X-Achse
		%    Beschriftung
		\bgblock{0}{1}{$B_4$}
		\bgblock{1}{4}{$B_1$}
		\bgblock{4}{5}{$B_4$}
		\bgblock{5}{7}{$B_3$}
		\bgblock{7}{8}{$B_2$}
		\bgblock{8}{9}{$B_4$}
		\bgblock{9}{12}{$B_1$}
		\bgblock{12}{13}{$B_4$}
		\bgblock{13}{15}{$B_3$}
		\bgblock{16}{17}{$B_4$}
		\bgblock{17}{20}{$B_1$}
		\bgblock{20}{21}{$B_2$}
		\bgblock{21}{22}{$B_4$}
		\bgblock{22}{24}{$B_3$}
		\bgblock{24}{27}{$B_1$}
	
	\end{blockgraph}
	
	\item
	todo
	\end{enumerate}
	
	\item
	todo
\end{enumerate}

\aufgabe{Prioriätsinversion}

\begin{enumerate}
	\item
	todo
\end{enumerate}

\end{document}
