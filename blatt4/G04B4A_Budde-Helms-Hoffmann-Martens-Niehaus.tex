\documentclass[ngerman]{fbi-aufgabenblatt}
\usepackage{amsmath}
\usepackage{amsfonts}
\usepackage{amssymb}
\usepackage{paralist}
\usepackage{blockgraph}

\usepackage{listings}
\usepackage{color}
\usepackage{textcomp}

\definecolor{mygreen}{rgb}{0,0.6,0}
\definecolor{mygray}{rgb}{0.5,0.5,0.5}
\definecolor{mymauve}{rgb}{0.58,0,0.82}

\lstset{ %
  backgroundcolor=\color{white},   % choose the background color
  basicstyle=\footnotesize,        % size of fonts used for the code
  breaklines=true,                 % automatic line breaking only at whitespace
  captionpos=b,                    % sets the caption-position to bottom
  commentstyle=\color{mygreen},    % comment style
  escapeinside={\%*}{*)},          % if you want to add LaTeX within your code
  keywordstyle=\color{blue},       % keyword style
  stringstyle=\color{mymauve},     % string literal style
}

% Folgende Angaben bitte Anpassen !!!

\renewcommand{\Aufgabenblatt}{4}
\renewcommand{\Gruppe}{G04}
\renewcommand{\KleinGruppe}{A}
\renewcommand{\Teilnehmer}{Budde, Helms, Hoffmann, Martens, Niehaus}

\begin{document}

\aufgabe{Scheduling-Algorithmen}

\begin{enumerate}

\item \color{white}.\color{black}\\
\begin{blockgraph}{30}{1}{0.4}
	% \bglabelxx erzeut Beschriftung der X-Achse an bestimmter Position
	\bglabelxx{0}
	\bglabelxx{5}
	\bglabelxx{10}
	\bglabelxx{15}
	\bglabelxx{20}
	\bglabelxx{25}
	\bglabelxx{30}

	% \bgblock erzeugt Block innerhalb des Graphen. Parameter:
	%    Y-Position (z.B. CPU), optional
	%    Beginn auf der X-Achse
	%    Ende auf der X-Achse
	%    Beschriftung
	\bgblock{1}{7}{$P_1$}
	\bgblock{8}{9}{$P_3$}
	\bgblock{10}{12}{$P_4$}
	\bgblock{13}{18}{$P_2$}
	\bgblock{19}{27}{$P_5$}

\end{blockgraph}

\item \color{white}.\color{black}\\
\begin{blockgraph}{33}{1}{0.4}
	% \bglabelxx erzeut Beschriftung der X-Achse an bestimmter Position
	\bglabelxx{0}
	\bglabelxx{5}
	\bglabelxx{10}
	\bglabelxx{15}
	\bglabelxx{20}
	\bglabelxx{25}
	\bglabelxx{30}

	% \bgblock erzeugt Block innerhalb des Graphen. Parameter:
	%    Y-Position (z.B. CPU), optional
	%    Beginn auf der X-Achse
	%    Ende auf der X-Achse
	%    Beschriftung
	\bgblock{1}{5}{$P_1$}
	\bgblock{6}{8}{$P_2$}
	\bgblock{9}{10}{$P_3$}
	\bgblock{11}{13}{$P_1$}
	\bgblock{14}{16}{$P_4$}
	\bgblock{17}{19}{$P_5$}
	\bgblock{20}{22}{$P_2$}
	\bgblock{23}{25}{$P_5$}
	\bgblock{26}{27}{$P_2$}
	\bgblock{28}{32}{$P_5$}
\end{blockgraph}

\item \color{white}.\color{black}\\
\begin{blockgraph}{31}{1}{0.4}
	% \bglabelxx erzeut Beschriftung der X-Achse an bestimmter Position
	\bglabelxx{0}
	\bglabelxx{5}
	\bglabelxx{10}
	\bglabelxx{15}
	\bglabelxx{20}
	\bglabelxx{25}
	\bglabelxx{30}
	% \bgblock erzeugt Block innerhalb des Graphen. Parameter:
	%    Y-Position (z.B. CPU), optional
	%    Beginn auf der X-Achse
	%    Ende auf der X-Achse
	%    Beschriftung
	\bgblock{1}{5}{$P_1$}
	\bgblock{6}{10}{$P_2$}
	\bgblock{11}{12}{$P_3$}
	\bgblock{13}{15}{$P_4$}
	\bgblock{16}{20}{$P_5$}
	\bgblock{21}{23}{$P_1$}
	\bgblock{24}{25}{$P_2$}
	\bgblock{26}{30}{$P_5$}

\end{blockgraph}

\item Antwortzeiten (AZ): \\
 \begin{tabular}{c||c|c|c|c|c|c}
  Aufgabe \textbackslash Prozess & $P_1$ & $P_2$ & $P_3$ & $P_4$ & $P_5$ & $\varnothing$ \\\hline\hline
    1a 				&  7   & 14 & 4 & 6  & 19 & 10 \\\hline
    1b 				&  13  & 23 & 4 & 10 & 24 & 14,8 \\\hline
    1c 				&  23  & 20 & 7 & 9  & 22 & 16,2 \\\hline
 \end{tabular}

\item Große $\Delta t$ haben den Vorteil das der Overhead im Vergleich zu
kleinen $\Delta t$ geringer ist, da seltener zwischen Prozessen gewechselt werden muss.
Kleine $\Delta t$ haben den Vorteil das Prozesse die nur wenige CPU-Zyklen benötigen
schneller bearbeitet werden, da die Antwortzeiten geringer sind.

\end{enumerate}

\aufgabe{Echtzeit \& Mehrprozessor-Scheduling}

\begin{enumerate}
	\item Für $i \in \{1,2,3\}$ sind die Aufträge $A_i$ mit Periodendauern $p_i = \{(4),(7),(3)\}$ und Bedienzeitforderungen $b_i = \{(1),(3),(1)\}$ gegeben. 
	Zu zeigen ist, dass es selbst mit einem idealen Scheduler (kein Overhead) nicht möglich ist alle Deadlines einzuhalten.
	Aus den Perioden $p_i$ ergibt sich, dass für die Einhaltung aller Deadlines ein periodischer Zyklus aus
	kgV\footnote{kleinstes gemeinsames vielfaches}$(7,4,3) = 84$ Takten existieren müsste. In diesen 84 Takten müsste
	$A_1$ 21 mal einen Takt, $A_2$ 12 mal drei Takte und $A_3$ 28 mal einen Takt belegen. Es sind also
	insgesamt $21 + 28 + (3\cdot 12) = 85$ Takte nötig. Dies ist offenkundig nicht möglich, unabhängig von der Qualität des Schedulers.
	\item
	\begin{enumerate}
	
	\item Earliest Deadline First:\\
	\begin{blockgraph}{28}{1}{0.4}
		% \bglabelxx erzeut Beschriftung der X-Achse an bestimmter Position
		\bglabelxx{0}
		\bglabelxx{4}
		\bglabelxx{8}
		\bglabelxx{12}
		\bglabelxx{16}
		\bglabelxx{20}
		\bglabelxx{24}
		\bglabelxx{28}

		% \bgblock erzeugt Block innerhalb des Graphen. Parameter:
		%    Y-Position (z.B. CPU), optional
		%    Beginn auf der X-Achse
		%    Ende auf der X-Achse
		%    Beschriftung
		\bgblock{0}{1}{$B_4$}
		\bgblock{1}{4}{$B_1$}
		\bgblock{4}{5}{$B_4$}
		\bgblock{5}{7}{$B_3$}
		\bgblock{7}{8}{$B_2$}
		\bgblock{8}{9}{$B_4$}
		\bgblock{9}{12}{$B_1$}
		\bgblock{12}{13}{$B_4$}
		\bgblock{13}{15}{$B_3$}
		\bgblock{16}{17}{$B_4$}
		\bgblock{17}{20}{$B_1$}
		\bgblock{20}{21}{$B_2$}
		\bgblock{21}{22}{$B_4$}
		\bgblock{22}{24}{$B_3$}
		\bgblock{24}{27}{$B_1$}
	
	\end{blockgraph}

 \newpage
 
	\item Rate Monotonic Scheduling (Je kürzer die Periodendauer desto höher die Priorität):\\
	\begin{blockgraph}{26}{1}{0.4}
		% \bglabelxx erzeut Beschriftung der X-Achse an bestimmter Position
		\bglabelxx{0}
		\bglabelxx{4}
		\bglabelxx{7}
		\bglabelxx{8}
		\bglabelxx{9}
		\bglabelxx{11}
		\bglabelxx{12}
		\bglabelxx{14}
		\bglabelxx{16}
		\bglabelxx{18}
		\bglabelxx{20}
		\bglabelxx{21}
		\bglabelxx{22}
		\bglabelxx{24}

		% \bgblock erzeugt Block innerhalb des Graphen. Parameter:
		%    Y-Position (z.B. CPU), optional
		%    Beginn auf der X-Achse
		%    Ende auf der X-Achse
		%    Beschriftung
		%---- B4 (1 Takt alle 4) hat höchste Prio ----
		\bgblock{0}{1}{$B_4$}
		\bgblock{4}{5}{$B_4$}
		\bgblock{8}{9}{$B_4$}
		\bgblock{12}{13}{$B_4$}
		\bgblock{16}{17}{$B_4$}
		\bgblock{20}{21}{$B_4$}
		\bgblock{24}{25}{$B_4$}
		%---- Wenn B4 bedient hat B1 (3 Takte alle 7) Prio ----
		\bgblock{1}{4}{$B_1$}
		\bgblock{7}{8}{$B_1$}
		\bgblock{9}{11}{$B_1$}
		\bgblock{14}{16}{$B_1$}
		\bgblock{17}{18}{$B_1$}
		\bgblock{21}{24}{$B_1$}
		%---- dann hat B3 (2 Takte alle 9) Prio ----
		\bgblock{5}{7}{$B_3$}
		\bgblock{11}{12}{$B_3$}
		\bgblock{13}{14}{$B_3$}
		\bgblock{18}{20}{$B_3$}
	
	\end{blockgraph}
	
	Wie leicht zu sehen ist wird $B_2$ nicht abgearbeitet. \\
	Die CPU Auslastung für die gegeben Werte ist $\frac 37 + \frac 1{11} + \frac 29 + \frac 14 \approx 99\%$\\
	Die kleinste obere Schranke bei 4 Aufträgen ist allerdings $4 \cdot \left(2^{\frac14}-1\right) \approx 0,76$. Es war
	daher auch nicht zu erwarten das ein Scheduling möglich ist.
	
	\end{enumerate}
	
	\item Load Sharing nach Prioritäten bei 7 Prozessen $P_1,\ldots,P_7$ auf 4 CPU-Kernen $CPU_0,\ldots,CPU_3$: \\
	\begin{blockgraph}{16}{4}{0.7}
		% \bglabelxx erzeut Beschriftung der X-Achse an bestimmter Position
		\bglabelxx{0}
		\bglabelxx{5}
		\bglabelxx{10}
		\bglabelxx{15}
		%\bglabely{Zeile}{Beschriftung} erzeugt die Beschriftung der Y-Achse, unterste Zeile ist {0}
		\bglabely{0}{$CPU_0$} 
		\bglabely{1}{$CPU_1$} 
		\bglabely{2}{$CPU_2$} 
		\bglabely{3}{$CPU_3$} 

		% \bgblock erzeugt Block innerhalb des Graphen. Parameter:
		%    Y-Position (z.B. CPU), optional
		%    Beginn auf der X-Achse
		%    Ende auf der X-Achse
		%    Beschriftung
		\bgblock[0]{0}{4}{$P_1$}
		\bgblock[1]{2}{7}{$P_3$}
		\bgblock[2]{2}{3}{$P_2$}
		\bgblock[3]{3}{11}{$P_4$}
		\bgblock[2]{3}{11}{$P_5$}
		\bgblock[0]{4}{5}{$P_2$}
		\bgblock[0]{5}{10}{$P_6$}
		\bgblock[1]{7}{11}{$P_2$}
		\bgblock[0]{10}{14}{$P_7$}
	\end{blockgraph}
\end{enumerate}

\aufgabe{Prioriätsinversion}

\begin{enumerate}
	\item Mars-Pathfinder-Mission:\\
	\begin{blockgraph}{52}{1}{0.3}
		% \bglabelxx erzeut Beschriftung der X-Achse an bestimmter Position
		\bglabelx{0}{000}
		\bglabelx{10}{010}
		\bglabelx{20}{020}
		\bglabelx{30}{030}
		\bglabelx{40}{040}
		\bglabelx{50}{050}
		%\bglabely{Zeile}{Beschriftung} erzeugt die Beschriftung der Y-Achse, unterste Zeile ist {0} 

		% \bgblock erzeugt Block innerhalb des Graphen. Parameter:
		%    Y-Position (z.B. CPU), optional
		%    Beginn auf der X-Achse
		%    Ende auf der X-Achse
		%    Beschriftung
		\bgblock{0}{10}{$B$}
		\bgblock{10}{30}{$Z$}
		\bgblock{30}{52}{$M$}
	\end{blockgraph}
		\begin{blockgraph}{52}{1}{0.3}
		% \bglabelxx erzeut Beschriftung der X-Achse an bestimmter Position
		\bglabelx{0}{050}
		\bglabelx{10}{060}
		\bglabelx{20}{070}
		\bglabelx{30}{080}
		\bglabelx{40}{090}
		\bglabelx{50}{100}
		%\bglabely{Zeile}{Beschriftung} erzeugt die Beschriftung der Y-Achse, unterste Zeile ist {0} 

		% \bgblock erzeugt Block innerhalb des Graphen. Parameter:
		%    Y-Position (z.B. CPU), optional
		%    Beginn auf der X-Achse
		%    Ende auf der X-Achse
		%    Beschriftung
		\bgblock{0}{10}{$M$}
		\bgblock{10}{20}{$B$}
		\bgblock{20}{30}{$Z$}
		\bgblock{30}{40}{$B$}
		\bgblock{40}{50}{$Z$}
	\end{blockgraph}
	\begin{blockgraph}{52}{1}{0.3}
		% \bglabelxx erzeut Beschriftung der X-Achse an bestimmter Position
		\bglabelx{0}{100}
		\bglabelx{10}{110}
		\bglabelx{20}{120}
		\bglabelx{30}{130}
		\bglabelx{40}{140}
		\bglabelx{50}{150}
		%\bglabely{Zeile}{Beschriftung} erzeugt die Beschriftung der Y-Achse, unterste Zeile ist {0} 

		% \bgblock erzeugt Block innerhalb des Graphen. Parameter:
		%    Y-Position (z.B. CPU), optional
		%    Beginn auf der X-Achse
		%    Ende auf der X-Achse
		%    Beschriftung
		\bgblock{15}{20}{$M$}
		\bgblock{20}{40}{$Z$}
		\bgblock{40}{52}{$M$}
	\end{blockgraph}
	\begin{blockgraph}{32}{1}{0.3}
		% \bglabelxx erzeut Beschriftung der X-Achse an bestimmter Position
		\bglabelx{0}{150}
		\bglabelx{10}{160}
		\bglabelx{20}{170}
		%\bglabely{Zeile}{Beschriftung} erzeugt die Beschriftung der Y-Achse, unterste Zeile ist {0} 

		% \bgblock erzeugt Block innerhalb des Graphen. Parameter:
		%    Y-Position (z.B. CPU), optional
		%    Beginn auf der X-Achse
		%    Ende auf der X-Achse
		%    Beschriftung
		\bgblock{0}{15}{$M$}
		\bgblock[1]{10}{32}{$W$}
	\end{blockgraph}\\
	Wobei $W$ den Watchdog darstellt der für einen Neustart sorgt, da $B$ in der Periode 120-160 nicht ausgeführt werden konnte.
	
	Dies liegt daran, dass $M$ die für $B$ nötigen Resourcen mittels Semaphore sperrt, aber von $Z$ aufgrund der höheren
	Priorität verdrängt wird. $B$ kann also $M$ nicht verdrängen. Weiter gibt $M$ aufgrund der Verdrängung durch $Z$
	die für $B$ nötigen Resourcen zu spät frei. Somit kann $B$ nicht mehr rechtzeitig ausgeführt werden.
	\item Prioriätsinversion beschreibt die Umkehrung von Prioritäten. Obwohl $B$ höhere Priorität als $M$ hat, muss $B$ auf die
	Freigabe von gemeinsam genutzten Resourcen warten. Diese Freigabe kann nur am Ende der Bedienzeit von $M$ erfolgen. Somit hat
	$M$ effektiv eine höhere Priorität als $B$.
	
	\newpage
	
	\item Die Lösung war eine Art Prioritätsvererbung. Wenn ein Task mit niedriger Priorität die selben Resourcen nutzt wie ein Task
	mit höherer Priorität, wird für beide die hohe Priorität vergeben.\footnote{vgl. https:\textfractionsolidus\textfractionsolidus research.microsoft.com\textfractionsolidus en-us\textfractionsolidus um\textfractionsolidus people\textfractionsolidus mbj\textfractionsolidus Mars\_Pathfinder\/Mars\_Pathfinder.html} 
	%https://research.microsoft.com/en-us/um/people/mbj/Mars_Pathfinder/Mars_Pathfinder.html
	Mit dieser Konfigurationsoption von VxWworks (dem verwendeten
	Betriebssystem), kann $M$ nicht mehr von $Z$ verdrängt werden und es ergibt sich folgender Ablauf (Auszug):\\
	
	\begin{blockgraph}{52}{1}{0.3}
		% \bglabelxx erzeut Beschriftung der X-Achse an bestimmter Position
		\bglabelx{0}{100}
		\bglabelx{10}{110}
		\bglabelx{20}{120}
		\bglabelx{30}{130}
		\bglabelx{40}{140}
		\bglabelx{50}{150}
		%\bglabely{Zeile}{Beschriftung} erzeugt die Beschriftung der Y-Achse, unterste Zeile ist {0} 

		% \bgblock erzeugt Block innerhalb des Graphen. Parameter:
		%    Y-Position (z.B. CPU), optional
		%    Beginn auf der X-Achse
		%    Ende auf der X-Achse
		%    Beschriftung
		\bgblock{15}{45}{$M$}
		\bgblock{45}{52}{$B$}
	\end{blockgraph}
	\begin{blockgraph}{32}{1}{0.3}
		% \bglabelxx erzeut Beschriftung der X-Achse an bestimmter Position
		\bglabelx{0}{150}
		\bglabelx{10}{160}
		\bglabelx{20}{170}
		%\bglabely{Zeile}{Beschriftung} erzeugt die Beschriftung der Y-Achse, unterste Zeile ist {0} 

		% \bgblock erzeugt Block innerhalb des Graphen. Parameter:
		%    Y-Position (z.B. CPU), optional
		%    Beginn auf der X-Achse
		%    Ende auf der X-Achse
		%    Beschriftung
		\bgblock{0}{5}{$B$}
		\bgblock{5}{10}{$Z$}
		\bgblock{10}{20}{$B$}
		\bgblock{20}{32}{$Z$}
	\end{blockgraph}
\end{enumerate}

\end{document}
